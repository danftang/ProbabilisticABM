\documentclass[letterpaper,10pt]{article}
\usepackage{epsfig,comment,amssymb,listings,amsmath}
\setlength{\parskip}{1.8mm}
\setlength{\parindent}{0mm}
\begin{document}
%\date{}
\title{\Large \bf Commutation relations in Fock space}

\author{
{\rm Daniel Tang}\\
}

\maketitle

% Use the following at camera-ready time to suppress page numbers.
% Comment it out when you first submit the paper for review.
%\thispagestyle{empty}

\section{Commutation identities}
%##########################################################

\begin{equation}
AB = BA + [A,B]
\end{equation}

\begin{equation}
[A, BC] = [A,B]C + B[A,C]
\end{equation}

\begin{equation}
[AB, C] =  A[B,C] + [A,C]B
\end{equation}

\begin{equation}
[A,B + C] = [A,B] + [A,C]
\label{sumcommute}
\end{equation}
\begin{equation}
[A,mB] = m[A,B]
\end{equation}
where $m$ is a scalar multiplicative constant.
\begin{equation}
[A,B] = -[B,A]
\end{equation}

\begin{equation}
[ab,AB] = a[b,A]B + [a,A]bB + Aa[b,B] + A[a,B]b
\end{equation}
\begin{equation}
 = a[b,A]B  + [a,A]Bb + aA[b,B] + A[a,B]b
\end{equation}

\begin{equation}
[A,B^n] = \sum_{q=1}^n B^{q-1}[A,B]B^{n-q}
\end{equation}

\begin{equation}
[A,B^{n+1}] = [A,B]B^n + [A,B^n]B - [[A,B^n],B]
\end{equation}

\begin{multline}
[[A,C],[B,D]]=[[[A,B],C],D]+[[[B,C],D],A]
\\+[[[C,D],A],B]+[[[D,A],B],C]
\end{multline}
(source Wikipedia "Commutator")

\begin{equation}
[[A,H],[B,H]]=[[[A,B],H],H]+[[[B,H],H],A]-[[[A,H],B],H]
\end{equation}
so
\begin{multline}
[[^nA,H],[^mB,H]]=[^2[[^{n-1}A,H],[^{m-1}B,H]],H]
\\+[[^{m+1}B,H],[^{n-1}A,H]]-[[[^nA,H],[^{m-1}B,H]],H]
\end{multline}


If we define $[^nA,H] = [\dots [[A,H],H] \dots ,H]$ to be the $n$-fold commutation:

\begin{equation}
[^n AB,H] = \sum_{m=0}^n {n \choose m}[^mA,H][^{n-m}B,H]
\end{equation}

\begin{equation}
[^n ABC,H] = \sum_{a+b+c = n} \frac{n!}{a!b!c!}[^aA,H][^bB,H][^cC,H]
\end{equation}

\section{Annihilation and creation operators}
%##########################################################

Beginning with

\[
[a_\psi,a_\phi^\dag] = a_\psi a_\phi^\dag - a_\phi^\dag a_\psi = 
\begin{cases}
 1 & \text{if }\psi=\phi \\
 0 & \text{otherwise} \\
 \end{cases}
\]


\section{$a^\dag$ and $a$ operators}
%##########################################################

\begin{equation}
[a^m, a^\dag] = ma^{m-1}
\end{equation}

\begin{equation}
[a,a^{\dag m}] = ma^{\dag(m-1)}
\end{equation}
the above holds for all m, even -ve.
\begin{equation}
[a^-,a^m] = ((a^- + a)^m - a^m)a^-
\end{equation}

\begin{equation}
[a^n,a^{\dag m}] = \sum_{q=1}^{\min(m,n)} \frac{m!n!}{q!(m-q)!(n-q)!} a^{\dag m-q}a^{n-q}
\end{equation}


\begin{equation}
[a^{\dag p}a^m, a^{\dag q}a^n] = 
a^{\dag p}[a^m, a^{\dag q}] a^n - 
a^{\dag q} [a^n, a^{\dag p}]a^m
\end{equation}

\section{$a^-$ operator}
%##########################################################

Define the $a^-$ operator such that
\[
a^-a^\dag = I
\]
where $I$ is the identity operator. Given this we can see immediately that $[a^\dag,a^-a^\dag] = 0$, so
\[
a^\dag a^-a^\dag -a^-a^\dag a^\dag = [a^\dag,a^-]a^\dag = 0
\]
So, for all states, $S$ other than the ground state,
\[
[a^\dag,a^-]S = 0
\]
For the ground state, $\emptyset$, in order to ensure $[a^\dag,a^-]\emptyset = 0$ we define
\[
a^\dag (a^-\emptyset) = \emptyset
\]
However, such terms as $a^-\emptyset$ will never arise through annihilation operators as they will always be multipllied by zero.

\section{$L_{ir}$ operator}
%##########################################################

Changes $\lambda_i$ to $(1-r)\lambda_i$

\begin{equation}
[L_{ir}, a_i^\dag] = 0
\end{equation}


\begin{equation}
[L_{ir}, a_i] = r\lambda_i L_{ir}
\end{equation}

\begin{equation}
[L_{ir}, a_i^n] =  L_{ir}(a_i^n - (a_i-r\lambda_i)^n)
\end{equation}

\section{$g_{ir}$ operator}
%##########################################################

Multiplies each basis by $(1-r)^{\Delta_i}$

\[
g_{ir}D_0 = 1
\]

\begin{equation}
[a_i^\dag, g_{ir}] = ra_i^\dag g_{ir}
\end{equation}

\begin{equation}
[a_i, g_{ir}] = \frac{r(\lambda - a_i)}{1-r}g_{ir}
\end{equation}
so
\begin{equation}
g_{ir}a_i^{\dag n} = (1-r)^na_i^{\dag n}g_{ir}
\end{equation}
and
\begin{equation}
[g_{ir}, a_i^{\dag n}] = ((1-r)^n - 1)a_i^{\dag n}g_{ir}
\end{equation}
also
\begin{equation}
g_{ir}a_i = \frac{a_i-r\lambda_i}{1-r}g_{ir}
\end{equation}
so
\begin{equation}
g_{ir}a_i^n = \left(\frac{a_i-r\lambda_i}{1-r}\right)^ng_{ir}
\end{equation}
\begin{equation}
g_{ir}a_i^n = (1-r)^{-n}\sum_{m=0}^n{n \choose m} a_i^m(-r\lambda_i)^{n-m}g_{ir}
\end{equation}

and
\begin{equation}
[g_{ir},a_i^n] = \left((1-r)^{-n}\sum_{m=0}^n{n \choose m} a_i^m(-r\lambda_i)^{n-m} - a_i^n\right) g_{ir}
\end{equation}
so
\begin{equation}
[g_{ir},a_i^{\dag n}a_i^m] = \left((1-r)^{n-m}\sum_{l=0}^m{m \choose l} a_i^{\dag n}a_i^l(-r\lambda_i)^{m-l} - a_i^{\dag n}a_i^m\right) g_{ir}
\end{equation}

\section{$Lg_{ir}$ operator}
%##########################################################

We can join the $L$ and $g$ operators into a compound operator that multiplies bases by $(1-r)^\Delta_i$ and multiplies $\lambda_i$ by $(1-r)$. It has the properties:

\begin{equation}
[Lg_{ir}, a_i^\dag] = -ra_i^\dag Lg_{ir}
\end{equation}


\begin{equation}
[Lg_{ir}, a_i] = \frac{ra_i}{1-r} Lg_{ir}
\end{equation}

So
\begin{equation}
Lg_{ir}a_i^{\dag n}a_i^m = (1-r)^{n-m}a_i^{\dag n}a_i^mLg_{ir}
\end{equation}
and
\begin{equation}
[Lg_{ir},a_i^{\dag n}a_i^m] = ((1-r)^{n-m} - 1)a_i^{\dag n}a_i^mLg_{ir}
\end{equation}



%{\footnotesize \bibliographystyle{acm}
%\bibliography{sample}}


%\theendnotes

\end{document}
